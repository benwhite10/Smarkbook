\documentclass[class=exam, crop=false, 12pt]{standalone}
\usepackage[T1]{fontenc}
\usepackage[oldstyle]{libertine}
\usepackage[libertine]{newtxmath}

\begin{document}
\subsection*{Arithmetic progressions}
For an arithmetic progression with first term $a$ and common difference $d$,
\[ x_{n} = a + (n-1)d\]
\[ S_{n} = \frac{n}{2}(x_{1} + x_{n}) = \frac{n}{2}[\,2a+(n-1)d\,] \]

\subsection*{Geometric progressions}
For a geometric progression with the first term $a$ and common ratio $r$,
\[ x_{n} = ar^{n-1} \]
\[ S_{n} = a \left( \frac{r^{n} -1}{r-1} \right) = a \left( \frac{1 - r^{n} }{1-r} \right) \]
\[ S_{\infty} = \frac{a}{1-r} \, , \, |r| < 1 \]

\subsection*{Quadratic equations}
The quadratic formula states that if $ax^{2}+bx+c = 0$ then $ x = \dfrac{-b \pm \sqrt{b^{2} - 4ac}}{2a}$.

\subsection*{Mensuration}
A sphere of radius $r$ has surface area $A = 4\pi r^{2}$ and volume $V = \frac{4}{3}\pi r^{3}$.\vspace{0.1in}\\ 
A cone of base radius $r$, slant height $l$ and perpendicular heigh $h$ has curved surface area $\pi rl$ and volume $V = \frac{1}{3}\pi r^{2}h$.\vspace{0.1in}\\
A sector of a circle of radius $r$ and having central angle $\theta$ radians has arc length $s=r\theta$ and sector area $A = \frac{1}{2}r^{2}\theta$.

\subsection*{Trigonometry}
The sine rule: $\dfrac{a}{\sin{A}} = \dfrac{b}{\sin{B}} = \dfrac{c}{\sin{C}}$\vspace{0.1in}\\
The cosine rule: $a^{2} = b^{2} + c^{2} - 2bc\cos{A}$\vspace{0.1in}\\
The first Pythagorean identity: $\cos^{2}{\theta} + \sin^{2}{\theta} \equiv 1$\vspace{0.1in}\\
The definition of the tangent function: $\tan{\theta} = \dfrac{\sin{\theta}}{\cos{\theta}}$

\subsection*{Logarithms}
The definition of a logarithm: $\log_{a}{b} = c \Leftrightarrow a^{c} = b$\vspace{0.1in}\\
The product-sum law: $\log_{a}{xy} = \log_{a}{x}+\log_{a}{y}$\vspace{0.1in}\\
The quotient-difference law: $\log_{a}{\dfrac{x}{y}} = \log_{a}{x} - \log_{a}{y}$\vspace{0.1in}\\
The logarithm of 1: $\log_{a}{1} = 0$ \vspace{0.1in}\\
The change-of-base law: $\log_{a}{b} = \dfrac{\log_{c}{b}}{\log_{c}{a}}$


\subsection*{Algebra and calculus}
The derivative of $x^{n} : \dfrac{\text{d}}{\text{d}x} x^{n} = nx^{n-1}$\vspace{0.1in}\\
The trapezium rule: $\int_{x_{1}}^{x_{n}}y \, \text{d}x \approx \frac{h}{2} \left[ y_{1} + 2(y_{2}+y_{3} + \dots + y_{n-1}) + y_{n} \right]$ where $h = 
\dfrac{x_{n} - x_{1}}{n}$

\end{document}
