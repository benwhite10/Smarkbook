\documentclass[class=exam, crop=false, 12pt]{standalone}
\usepackage[T1]{fontenc}
\usepackage[oldstyle]{libertine}
\usepackage[libertine]{newtxmath}


\begin{document}
\glossitem{acute angle}{An angle smaller than a \emph{right angle}.}
\glossitem{arithmetic mean}{The sum of a list of numbers divided by the number of elements in the list.}
\glossitem{axle}{The rod at the centre of the wheel; the fixed point about which it rotates.}
\glossitem{bisect}{To divide (a line or region) into two equal parts; to halve.}
\glossitem{chord}{A straight line joining two points on the circumference of a circle.}
\glossitem{circle}{A set of points equidistant from another point, the centre.}
\glossitem{coefficient}{The number by which a \emph{term} has been multiplied. e.g. the coefficient of $x^3$ in $4x^6 - 2x^3 + 3$ is $-2$}
\glossitem{conjecture}{A proposition, or mathematical statement, which may not yet have been proved to be true.}
\glossitem{convergent}{A sequence where the terms get closer to a fixed value, known as the \emph{limit}.}
\glossitem{cuboid}{A solid with six faces, each of which is a rectangle.}
\glossitem{decreasing function}{A function whose value decreases as the input increases; so $\fx$ is decreasing if $\fdx < 0$. A function may be classified as decreasing over a subset of its domain.}
\glossitem{denominator}{The bottom line of a fraction; the denominator of $\dfrac{p}{q}$ is $q$.}
\glossitem{derivative}{An expression representing the rate of change of a function with respect to an independent variable. We define $\fdx$, the derivative of a function $\fx$, as \\$\fdx = \lim_{h\rightarrow0}\left(\dfrac{\fx[x+h] - \fx}{h}\right)$. The derivative of y with respect to x is denoted $\dfrac{\mathrm{d}y}{\mathrm{d}x}$. See \emph{differential}.}
\glossitem{difference}{Inconsistently defined. In most uses, the difference of $x$ and $y$ is given by $|x-y|$, i.e. whichever of $x-y$ or $y-x$ is positive.}
\glossitem{differential}{A car part; specifically a gear allowing a vehicle's wheels to turn at different speeds to allow cornering. See \emph{derivative}.}
\glossitem{discriminant}{For a quadratic equation of the form: \[ax^{2} + bx + c = 0, a \neq 0\] the discriminant, $\Delta$, is given by $\Delta=b^2-4ac$.}
\glossitem{divergent}{Not \emph{convergent}.}
\glossitem{divisor}{Something that divides exactly into something else. A divisor of $x$ is something which goes into $x$ exactly; 3 is a divisor of 24. See \emph{factor}.}
\glossitem{domain}{The set of ``input values'' for a function.}
\glossitem{drag car}{A vehicle that takes part in a drag race, where (usually two) cars race side-by-side on a long straight track. In commercial drag racing, cars start from rest and race on tracks that are $\frac{1}{4}$ mile long (402 metres).}
\glossitem{equilateral}{Having all sides the same length; e.g. an equilateral triangle.}
\glossitem{expression}{Something that represents a value, e.g. $3x^2+4k^3xy$ is an expression because it represents a value; in this case the value depends on the values of $k, x$ and $y$.}
\glossitem{face}{Any one of the 2D surfaces of a 3D shape.}
\glossitem{factor}{A divisor of a number or expression. In $5x(3+2x)$ the factors are $5, x$ and $(3+2x)$.}
\glossitem{factors of $x$}{Usually taken to mean all the different positive integers which are factors of $x$; e.g. the factors of 12 are 1, 2, 3, 4, 6 and 12.}
\glossitem{frequency density}{The height of a rectangle in a histogram.}
\glossitem{HCF}{Highest Common Factor. HCF$(x,y)$: The largest integer which divides exactly into both $x$ and $y$.}
\glossitem{hexagon}{A shape with six straight edges, not necessarily regular.}
\glossitem{hypothesis}{A statement consistent with known data, but which has not yet been shown to be true or false; alternatively, a supposition upon which further steps of reasoning may be based, such as an \emph{induction hypothesis} in logic or a \emph{null hypothesis} in statistics.}
\glossitem{identity}{A statement that two expressions are equal in value, regardless of the values of any unknowns in the statement, e.g. $4(x+2) \equiv 4x+8$. Alternatively, the operation or transformation that leaves an object unchanged.}
\glossitem{increasing function}{A function whose value increases; so $\fx$ is increasing if $\fdx > 0$. A function may be classified as increasing over a subset of its domain.}
\glossitem{index}{Also known as an exponent or power. The index of $x^4$ is 4.}
\glossitem{instantaneous}{At the precise moment; not averaged over a period of time.}
\glossitem{integer}{A whole number, positive or negative. A member of the set $\mathbb{Z}$}
\glossitem{interquartile range}{The difference between the \emph{upper quartile} and the \emph{lower quartile}.}
\glossitem{irregular polygon}{Shape where not all angles or not all lengths are equal.}
\glossitem{isosceles}{Triangle with two sides of equal length.\\ The \emph{pons asinorum} tells us that it also possesses two equal angles.}
\glossitem{kite}{Quadrilateral with two pairs of sides of equal length; the equal-length sides meet at a vertex.}
\glossitem{LCM}{Lowest (least) common multiple. LCM$(x,y)$: the smallest integer which $x$ and $y$ both divide exactly into.}
\glossitem{line segment}{The parts of a line falling between two points.}
\glossitem{line}{A 1D set of points which goes on for ever in both directions, such that for any two points in the set, the shortest route between those two points lies entirely on the line. See \emph{segment}.}
\glossitem{locus}{The set of all points which satisfy an algebraic or geometric rule.}
\glossitem{logarithm}{$\log_{a}c=b \Leftrightarrow a^b=c$.}
\glossitem{lower quartile}{See \emph{quartile}.}
\glossitem{major arc}{The long way around the circumference of a circle given two boundary points.}
\glossitem{mathematician}{``A machine for converting coffee into theorems'' (Alfred R$\acute{\mathrm{e}}$nyi).}
\glossitem{mean}{Any of a number of measures of average. Usually the \emph{arithmetic mean} is intended.}
\glossitem{median}{The middle value, or arithmetic mean of the two middle values, of a list of data sorted in ascending order.}
\glossitem{minor arc}{The short way around the circumference of a circle given two boundary points.}
\glossitem{modal class}{For any set of grouped data, the class or interval with the highest frequency density.}
\glossitem{mode}{The most frequently occurring value in a dataset.}
\glossitem{multiple of $x$}{Something of which $x$ is a factor.}
\glossitem{normal}{Perpendicular to; the line intersecting a curve or curved surface that is perpendicular to the tangent at that point.}
\glossitem{numerator}{The top line of a fraction; the numerator of $\dfrac{p}{q}$ is $p$.}
\glossitem{obtuse angle}{An angle $x$ that is between a right angle and a half turn, i.e. $90\degree<x<180\degree$.}
\glossitem{octagon}{A shape with eight straight sides, not necessarily regular. Its internal angle is $1080\degree$.}
\glossitem{parabola}{A conic section formed by slicing parallel to the slant of the cone; the characteristic of the graph of a quadratic function.}
\glossitem{parallel}{Two lines are parallel if they never intersect.}
\glossitem{parallelogram}{A quadrilateral with two pairs of parallel sides.}
\glossitem{pentagon}{A shape with five straight sides, not necessarily regular. Its internal angle is $540\degree$.}
\glossitem{perimeter}{The sum of the lengths of the edges of a 2D shape.}
\glossitem{period}{The interval before something repeats itself.}
\glossitem{periodic}{Repeating. Periodic sequences repeat after a certain number of terms. The number of terms before the sequence repeats is known as the \emph{period}.}
\glossitem{perpendicular bisector}{The perpendicular bisector of a line PQ passes through the midpoint of PQ at right angles to PQ.}
\glossitem{percentage error}{The percentage difference between an approximate and an exact value, measured in terms of the exact value. For an approximation $A$ to an exact value $a$, the percentage error is given by $\Big|\dfrac{a-A}{a}\Big|\times100\%$.}
\glossitem{perpendicular}{At right angles to. Perpendicular lines have gradients $m_1$ and $m_2$ such that $m_1m_2 = -1$. See \emph{Normal}.}
\glossitem{plane}{A 2D region or flat surface.}
\glossitem{plane figure}{A shape lying entirely within a plane, i.e. a 2D shape.}
\glossitem{polygon}{Any shape with only straight sides.}
\glossitem{power}{An \emph{index} or \emph{exponent}, or the result of such operation: the power of $x$ in $x^4$ is 4, also 16 is the fourth power of 2; 16 is a power of 2.}
\glossitem{prime}{A positive integer whose only factors are itself and 1.}
\glossitem{prism}{A solid with a constant cross section; the result of an extrustion.}
\glossitem{product of $x$ and $y$}{$x\times y$}
\glossitem{pronumeral}{Any letter used to take the place of a number in an algebraic context.}
\glossitem{prove}{To provide a rigorous argument justifying a theorem unequivocally. A well-written proof should need no further explanation. See \emph{theorem} and \emph{conjecture}.}
\glossitem{quadratic}{An expression, function or equation in which the highest power is 2; i.e. in the form $ax^2+bx+c$.}
\glossitem{quadrilateral}{Any plane figure with four sides.} 
\glossitem{quartile}{The \emph{lower quartile} and \emph{upper quartile} are the values which are 25\% and 75\% of the way through a dataset arranged in ascending order.}
\glossitem{quidditch}{A fictional sport from the Harry Potter series by J. K. Rowling, played by opposing teams of seven players, each having three chasers, two beaters, one keeper and one seeker.}
\glossitem{quotient of $x$ over $y$}{$\dfrac{x}{y}$}
\glossitem{range}{(Statistics) The difference between the maximum and minimum values in a dataset or distribution;\\ (Functions) The set of output values of a function.}
\glossitem{rational}{A number which can be written in the form $\dfrac{p}{q}$, where $p,q$ $\in \mathbb{R}, q \neq 0$.}
\glossitem{reciprocal of $x$}{$\dfrac{1}{x}$}
\glossitem{recurring decimal}{A number whose decimal expansion consists (after a certain point) only of repeating strings which go on for ever. Terminating decimals end with a recurring 0, so are recurring decimals.}
\glossitem{reflex angle}{An angle greater than a half turn but smaller than a full turn; i.e. between $180\degree$ and $360\degree$.}
\glossitem{regular polygon}{A straight-sided plane figure (\emph{polygon}) whose angles are all equal (\emph{equiangular}) and edges are all equal (\emph{equilateral})}
\glossitem{right angle}{An angle of $90\degree$; a quarter turn.}
\glossitem{rhombus}{A quadrilateral with four sides of equal length.\\ ``A square which, after a rough night of drinking, is \dots unable to stand up straight.'' -- \emph{Urban Dictionary}}
\glossitem{scalene}{A triangle in which no two sides are of equal length.}
\glossitem{sector}{The part of a circle bounded by two radii and an arc of a circle; a pizza slice. See \emph{segment}.}
\glossitem{segment}{The part of a circle bounded by a chord and an arc of the circle; a pizza slice guaranteed to draw incredulity from your friends. See \emph{sector}.}
\glossitem{sequence}{An ordered list of numbers. Each number in the sequence is known as a \emph{term}.}
\glossitem{subtends}{To span an arc or line segment; an angle subtends an arc; e.g. an angle of $120\degree$ at the centre of a circle subtends one-third of the circumference.}
\glossitem{sum of $x$ and $y$}{$x+y$}
\glossitem{surd}{An irrational number expressed exactly, usually a square root: e.g. $\sqrt{2}$ or $\big(2 + \sqrt{5}\big)$ but not 1.414 or 4.236 to represent the same quantities.}
\glossitem{tangent}{A straight line or plane that touches a curve or curved surface at a point, but---if extended---does not cross it at that point; a line which touches a circle exactly once.}
\glossitem{term}{Factorised expressions which are added together to form a longer expression; in $3x^2+17ax-5$, the terms are $3x^2$, $17ax$ and $-5$. Also, a constant element of a \emph{sequence}.}
\glossitem{terminating decimal}{A decimal expansion that stops after a finite number of digits (i.e. all the successive digits are 0).}
\glossitem{tetrahedron}{A solid shape with four equilateral triangle as faces; a regular triangular-based pyramid.}
\glossitem{theorem}{A proposition, or mathematical statement, which has been proven to be true. See \emph{conjecture}.}
\glossitem{trajectory}{The path taken by an object as it moves through space; typically an object in flight.}
\glossitem{trapezium}{A quadrilateral with one pair of parallel sides, unless you think footballs are pointy and babies wear diapers, in which case it is a quadrilateral with no parallel sides.}
\glossitem{trapezoid}{A quadrilateral with no pair of parallel sides, unless you think cars drive on the pavement and jelly is a sandwich filling in which case it is a quadrilateral with one pair of parallel sides.}
\glossitem{upper quartile}{See \emph{quartile}.}
\glossitem{variable}{A \emph{pronumeral} which may take different values within the context of a problem.}
\glossitem{vertex}{The corner of a shape. Plural: vertices.}
\end{document}
